\documentclass{homework}
% \author{Tashfeen, Ahmad}  % uncomment with your name
\class{CSCI 3203: Tashfeen's Machine Learning I}
\title{Homework 2}
\address{
  Oklahoma City University,
  Petree College of Arts \& Sciences,
  Computer Science
}

\begin{document} \maketitle

You're only allowed to use the standard built-in Python 3
libraries with the exception of Numpy and Matplotlib.

\question Plot the following trigonometric functions for $-5\pi \leq x \leq 5\pi$.
\[
  f(x) = x\sin^2(x), \quad  g(x) = -x\sin^2(x)
\]
\begin{itemize}
  \item The title of the plot should read ``Trigonometric Functions.''
  \item The $x$--axis label of the plot should read ``$x$--axis.''
  \item The $y$--axis label of the plot should read ``$y$--axis.''
  \item The legends should appropriately denote the different functions by their definition.
  \item The graphs should have smooth continuos curves.
\end{itemize}

\question Download and save the novel
\href{https://www.gutenberg.org/files/20727/20727.txt}{``The Cosmic
Computer''} by Piper. Read the saved text file into Python and plot
a bar graph of the character frequency (case insensitive).

\begin{itemize}
  \item The title of the plot should read `Character Frequency in ``The Cosmic Computer'' by Piper.'
  \item The $x$--axis label of the plot should read ``Alphabet.''
  \item The $y$--axis label of the plot should read ``Frequency.''
  \item The bars should be sorted in ascending order.
\end{itemize}

\question Using \texttt{matplotlib.pyplot.imshow} and an eight by eight binary Numpy array, plot the chess board pattern.

\begin{itemize}
  \item The $x$--ticks should be letters from A to H, A being at the left.
  \item The $y$--ticks should be numbers from 1 to 8, one being at the bottom.
  \item The title should be ``Chess Board Pattern.''
\end{itemize}

\question Following are the two matrices from the last homework,
\[
  \mathbf{A} =
  \begin{bmatrix} 1 & 0 & 1 \\   2 & 1 & 1 \\   0 & 1 & 1 \\   1 & 1 & 2 \\
  \end{bmatrix}, \quad
  \mathbf{B} =
  \begin{bmatrix} 1 & 2 & 1 \\   2 & 3 & 1 \\   4 & 2 & 2 \\
  \end{bmatrix}
\]
Use Numpy to multiply them and print the product $\mathbf{AB}$. Does it match with the one you carried out by hand?

\begin{sol}
  Yes.
\end{sol}

\question The Gregory series is given as,
\[
  f(n) = \sum_{i=1}^{i=n}\frac{4(-1)^{i+1}}{2i-1}=\frac{4}{1}-\frac{4}{3}+\frac{4}{5}-\cdots+\frac{4(-1)^{n+1}}{2n-1}
\]
\begin{enumerate}
  \item Without using any loops \ie using vectorised operations over Numpy arrays, plot the following values,
    \[
      \{(n, f(n)) : 1 \leq n \leq 100\}
    \]
  \item Plot the following error curve,
    \[
      \{(n, (\pi - f(n))^2) : 1 \leq n \leq 100\}
    \]
\end{enumerate}
Label the axis and title each plot appropriately. What is $\lim_{n\ra\infty}f(n)$?

% solution goes here
\begin{sol}
  $\pi$
\end{sol}

\question Consider the following study results in table \ref{labels},

\tbl<labels>{Real and Predicted Labels}{
  Point & Real  & Predicted \\
  1     & $-$     & $+$         \\
  2     & $-$     & $-$         \\
  3     & $-$     & $-$         \\
  4     & $-$     & $+$         \\
  5     & $+$     & $-$         \\
  6     & $+$     & $+$         \\
  7     & $+$     & $+$         \\
  8     & $+$     & $+$         \\
}

State all the points that are,

\begin{enumerate}
  \item True Negatives
    % solution goes here
    \begin{sol}
      2, 3
    \end{sol}
  \item True Positives
    % solution goes here
    \begin{sol}
      6, 7, 8
    \end{sol}
  \item False Positives
    % solution goes here
    \begin{sol}
      1, 4
    \end{sol}
  \item False Negatives
    % solution goes here
    \begin{sol}
      5
    \end{sol}
\end{enumerate}

\question Give the confusion matrix for table \ref{labels}.
\tbl<mat>{Binary Confustion Matrix}{
  & Predicted $-$ & Predicted $+$  \\
  Real $-$    & TN            & FP         \\
  Real $+$    & FN            & TP         \\
}

% solution goes here
\begin{sol}
  \tbl<mat1>{Binary Confustion Matrix}{
    & Predicted $-$ & Predicted $+$  \\
    Real $-$& 2&2\\
    Real $+$& 1&3\\
  }
\end{sol}

\question Again, for table \ref{labels}, give the following in exact fractions.
\begin{enumerate}
  \item Baseline Accuracy
  \item Accuracy
    \[
      \frac{\text{Total Correctly Classified}}{\text{Total Examples}}
    \]
  \item Precision
    \[
      \frac{\text{True Positives}}{\text{Total Examples Classified as Positive}}
    \]
  \item Recall
    \[
      \frac{\text{True Positives}}{\text{True Positives} + \text{False Negatives}}
    \]
  \item F1 Score
    \[
      \frac{2\text{TP}}{2\text{TP} + \text{FP} + \text{FN}}
    \]
\end{enumerate}

% solution goes here
\begin{sol}
  \begin{itemize}
    \item $= \frac{5}{8}$
    \item $= \frac{3}{5}$
    \item $= \frac{3}{4}$
    \item $= \frac{2(3)}{2(3)+2+1} = \frac{6}{9} = \frac{2}{3}$
  \end{itemize}
\end{sol}

\question Often (but not always) how are precision and recall related?

% solution goes here
\begin{sol}
  Precision and recall can often be inverse to each other. If you counted everything as a positive, your precision would be down but your recall would be maximized.
  however the 2 measurements are not directly inverse to each other.
\end{sol}

\question Between precision and recall, which is a better metric for,
\begin{enumerate}
  \item Predicting good restaurants.
  \item Predicting cancer.
\end{enumerate}

% solution goes here
\begin{sol}
  Precision is better of predicting good restaurants and recall is better for predicting cancer.
\end{sol}

\question Between precision and recall, which metric is used in the Receiver Operator's Curve (ROC). What does the area under this curve tell us about a prospective classifier?

% solution goes here
\begin{sol}
  The ROC uses recall. The area under the curve tells us the recall rate of the prospective classifier.
\end{sol}

\section*{Submission Instructions}

\begin{enumerate}
  \item Submit a PDF that answers all the questions and includes any plots that the assignment asks for. Circle and/emphasize your final answer wherever possible.
  \item Submit a Python file \eg \texttt{plot.py} that generates all the plots one by one.
\end{enumerate}

\img<img1>[0.65]{Question 1 Plot}{./media/hw2q1}
\img<img2>[0.65]{Question 2 Plot}{./media/hw2q2}
\img<img3>[0.65]{Question 3 Plot}{./media/hw2q3}
\img<img4>[0.65]{Question 5 Plot}{./media/hw2q5}
\end{document}
