\documentclass{homework}
\author{Morales, Samantha}  % uncomment with your name
\class{CSCI 3203: Tashfeen's Machine Learning I}
\title{Homework 1}
\address{
  Oklahoma City University,
  Petree College of Arts \& Sciences,
  Computer Science
}

\DeclareMathOperator{\row}{row}
\DeclareMathOperator{\col}{col}

\newcommand\marry{\href{%
    https://www.youtube.com/watch?v=mGYmiQkah4o%
  }{Mary's Room: A philosophical thought experiment -- Eleanor Nelsen}}

\newcommand\listcomp{\href{%
    https://www.w3schools.com/python/python_lists_comprehension.asp%
  }{Python list comprehension}}

\begin{document} \maketitle

Please show all work for full credit. You're only allowed to use
the standard built-in Python 3 libraries.

\question Watch \marry{} on YouTube.

Do you think there is something about human learning that is
different than machine learning?

\begin{sol}
  % solution goes here
  Not intrinsically, I believe that human learning relies heavily on perception and up until now I don't think we've really implemented machine learning 
  algorithms that can truly perceive and learn in the same way humans can; however, I don't think this would be impossible to do. 
\end{sol}

\question Rate your skills out of ten in the following subjects,
\begin{enumerate}
  \item Probability and Statistics 
  \begin{sol}
    - 5
  \end{sol}
  \item Calculus 
  \begin{sol}
    - 5
    \end{sol}
  \item Linear Algebra 
  \begin{sol}
    - 5
  \end{sol}
  \item Python
  \begin{sol}
    - 7
  \end{sol}
\end{enumerate}

\question Of the sub-plots (a), (b), (c) and (d) in figure ??, state
which ones have no correlation, negative correlation, positive
correlation and non-linear correlation.

% \img<cor>[0.5]{Different types of correlations.}{media/correlation.png}

\begin{sol}
  % solution goes here
  C, B, A, D; Respectively
\end{sol}

\question Assume that the probability of Alice going to class everyday given
she got a good grade is $\frac{5}{6}$ and the probability of her
getting a good grade is $\frac{2}{5}$ while the probability of her
going to class everyday is $\frac{1}{3}$. What is the probability
that Alice gets a good grade given she went to class everyday?

Hint: Thomas Bayes.

\begin{sol}
  % solution goes here
  \[
  P(B|A) = \frac{5}{6}, P(A)=\frac{2}{5}, P(B)=\frac{1}{3} 
  \]
  \[
  P(A|B) = \frac{P(B|A)P(A)}{P(B)} = \frac{\frac{5}{6} \times \frac{2}{5}}{\frac{1}{3}} = \frac{\frac{1}{3}}{\frac{1}{3}} = 1
  \]
\end{sol}

\question We define the natural numbers as,
\[
  \N = \{1, 2, 3, 4, 5, ...\}
\]
Let $x$ be the arithmetic mean of the first $2^{33}-1$ natural
numbers. What is $\log_2(x)$?

Hint: Carl Friedrich Gauss.

\begin{sol}
  % solution goes here
  \[
  \frac{\frac{n(n+1)}{2}}{n} = \frac{n(n+1)}{2n} = \frac{n+1}{2} = \frac{2^{33} - 1 + 1}{2} = 2^{32} => \log_2(2^{32}) = 32
  \]
\end{sol}

\question Let \texttt{arr = [x \% 2 for x in range((2**100)-1)]} be a
Python list using \listcomp{}. What is the statistical mode of the
list \texttt{arr}? Justify your answer.

\begin{sol}
  % solution goes here
  0
\end{sol}

\question Define a function $f:\R\ra\R$,
\[
  f(x) = \frac{\sin(x^2)}{y}
\]
Calculate the following,
\begin{itemize}
  \item The partial derivative with respect to $y, \displaystyle\frac{\p
            f}{\p y}$
        % solution goes here
        \begin{sol}
          \[
          -\frac{\sin(x^2)}{y^{2}}
          \]
        \end{sol}
  \item The partial derivative with respect to $x, \displaystyle\frac{\p
            f}{\p x}$
        % solution goes here
        \begin{sol}
          \[
          \frac{2x\cos(x^2)}{y}
          \]
        \end{sol}
  \item The gradient vector $\nabla f(x, y)$.
        % solution goes here
        \begin{sol}
          \[
          [\frac{2x\cos(x^2)}{y},-\frac{\sin(x^2)}{y^{2}}]
          \]
        \end{sol}
\end{itemize}

\question Let $\vec{v_1} = [e, \pi, \sqrt{2}]$ and $\vec{v_2} = [1, 2, 0]$.
What is the dot product $\vec{v_1} \cdot \vec{v_2}$? Please do not
give an approximation.

\begin{sol}
  % solution goes here
  $e + 2 \pi$
\end{sol}

\question We define a matrix of $n$ rows and $p$ columns with real values as
$\mathbf{A} \in \R^{n\times p}$. We can think of $\mathbf{A}$ as
having $n$ row vectors with the $i^\text{th} \leq n$ row vector
being $\row(\mathbf{A})_i$. Similarly, we can think of
$\mathbf{A}$ as having $p$ column vectors with the $j^\text{th}
  \leq p$ column vector being $\col(\mathbf{X})_j$.

Then matrix multiplication between $\mathbf{A} \in \R^{n\times p}$
and $\mathbf{B} \in \R^{p\times m}$ is defined as,
\[
  \mathbf{AB} =
  \begin{bmatrix}
    \row(\mathbf{A})_1\col(\mathbf{X})_1 & \row(\mathbf{A})_1\col(\mathbf{X})_2 & \cdots & \row(\mathbf{A})_1\col(\mathbf{X})_m \\
    \row(\mathbf{A})_2\col(\mathbf{X})_1 & \row(\mathbf{A})_2\col(\mathbf{X})_2 & \cdots & \row(\mathbf{A})_2\col(\mathbf{X})_m \\
    \vdots                               & \vdots                               & \ddots & \vdots                               \\
    \row(\mathbf{A})_n\col(\mathbf{X})_1 & \row(\mathbf{A})_n\col(\mathbf{X})_2 & \cdots & \row(\mathbf{A})_n\col(\mathbf{X})_m \\
  \end{bmatrix}
\]
where $\row(\mathbf{A})_i\col(\mathbf{X})_j \in \R$ is a dot or
inner product.

Further, the transpose of $\mathbf{A} \in \R^{n\times p}$ is
written as $\mathbf{A}^T \in \R^{p\times n}$ and is defined as,
\[
  \text{If } \mathbf{A}^T \text{ is the transpose of } \mathbf{A} \text{ then } \row(\mathbf{A})_i = \col(\mathbf{A}^T)_j.
\]
\begin{enumerate}[label=(\alph*)]
  \item What must be true of the number of columns of $\mathbf{A}$ and
        number of rows of $\mathbf{B}$ for $\mathbf{AB}$ to be defined?

        % solution goes here
        \begin{sol}
          The number of columns in A must match the number of rows in B.
        \end{sol}

  \item What information do the number of rows of $\mathbf{A}$ and number
        of columns of $\mathbf{B}$ give you about the dimensions of the
        product $\mathbf{AB}$?

        % solution goes here
        \begin{sol}
          The number of rows in A and the number of columns in B give us the
          number of rows and columns in AB (respectively).
        \end{sol}

  \item For the two matrices bellow, give the product $\mathbf{AB}$.
        \[
          \mathbf{A} = \begin{bmatrix} 1 & 0 & 1 \\   2 & 1 & 1 \\   0 & 1 & 1 \\   1 & 1 & 2 \\   \end{bmatrix}, \quad
          \mathbf{B} = \begin{bmatrix} 1 & 2 & 1 \\   2 & 3 & 1 \\   4 & 2 & 2 \\   \end{bmatrix}
        \]
        % solution goes here
        \begin{sol}
          \[
          \mathbf{AB} = \begin{bmatrix}
            5&4&3\\
            8&9&5\\
            6&5&3\\
            11&9&6\\
          \end{bmatrix}
          \]
        \end{sol}

  \item Prove that the matrix product is not a symmetric relation \ie
        $\mathbf{AB} \neq \mathbf{BA}$. Hint: can you construct a small
        counter example?
        % solution goes here
        \begin{sol}
          BA is not even possible because the number of columns in B does not match the number of rows in B.
        \end{sol}
  \item Prove that $\mathbf{(AB)}^T = \mathbf{B}^T\mathbf{A}^T$.
        % solution goes here
        \begin{sol}
          % \[
          % \mathbf{B}^T = 
          % \begin{bmatrix}
          %   1&2&4\\
          %   2&3&2\\
          %   1&1&2\\
          % \end{bmatrix}, \quad
          % \mathbf{A}^T =
          % \begin{bmatrix}
          %   1&2&0&1\\
          %   0&1&1&1\\
          %   1&1&1&2\\
          % \end{bmatrix} 
          % \]
          % \[
          % \mathbf{B}^T\mathbf{A}^T = 
          % \begin{bmatrix}
          %   5&8&6&11\\
          %   4&9&5&9\\
          %   3&5&3&6\\
          % \end{bmatrix}, \quad
          % \mathbf{(AB)}^T =
          % \begin{bmatrix}
          %   5&8&6&11\\
          %   4&9&5&9\\
          %   3&5&3&6\\
          % \end{bmatrix}
          % \]
          % QED

          \begin{proof} $\mathbf{(AB)}^T = \mathbf{B}^T\mathbf{A}^T$
            \begin{enumerate}[label=(\arabic*)]
              \item $(\mathbf{AB})_{j,i}^T = \mathbf{AB}_{i,j} = \row(\mathbf{A})_i \col(\mathbf{B})_j$
              \item $(\mathbf{B}^T\mathbf{A}^T)_{j,i} = \row(\mathbf{B}^T)_j \col(\mathbf{A}^T)_i = \row(\mathbf{A})_i \col(\mathbf{B})_j$
              \item therefore, $\mathbf{(AB)}^T = \mathbf{B}^T \mathbf{A}^T$
            \end{enumerate}
          \end{proof}



        \end{sol}

  \item For $\mathbf{X} \in \R^{n\times p}, \beta \in \R^p, y \in \R^n$
        prove that,
        \[
          \beta^T\mathbf{X}^Ty = y^T\mathbf{X}\beta
        \]
        % solution goes here
        \begin{sol}
          \begin{proof} $\beta^T\mathbf{X}^Ty = y^T\mathbf{X}\beta$
            \begin{enumerate}[label=(\arabic*)]
              \item $\beta^T\mathbf{X}^Ty$ and $y^T\mathbf{X}\beta$ return a 1x1 matrix
              \item we can perform a transpose on either of them since and 1x1 matrix equals itself when transposed         
              \item $(\beta^T\mathbf{X}^Ty)^T = y^T(\beta^T\mathbf{X}^T)^T$
              \item $y^T(\beta^T\mathbf{X}^T)^T = y^T\mathbf{X}\beta$
              \item therefore, $\beta^T\mathbf{X}^Ty = y^T\mathbf{X}\beta$
            \end{enumerate}
          \end{proof}
        \end{sol}
        
  \item Let $f(\beta) = \beta^T\beta$. Prove that the derivative $\frac{\D
            f}{\D \beta} = 2\beta$.
        % solution goes here
        \begin{sol}
          \begin{proof} $\frac{\D
            f}{\D \beta} = 2\beta$, for $f(\beta) = \beta^T\beta$
            \begin{enumerate}[label=(\arabic*)]
              \item Assuming the definition for $\beta$ as defined in part f, $\beta \in \R^p$ 
              \item $\col(\mathbf{\beta})_i = \row(\mathbf{\beta}^T)_i$
              \item since the values of $\beta$ are being muliplied with themselves, this means $\beta^T\beta = \langle \beta^2_1 + \beta^2_2... + \beta^2_i \rangle$
              \item since we have to take the partial dervative for each $\beta_i$ every time, we are left with only the $\beta_i$ since the rest are considered constants
              \item this leaves us with $2\beta$
            \item therefore, $\frac{\D
            f}{\D \beta} = 2\beta$, for $f(\beta) = \beta^T\beta$
            \end{enumerate} 
          \end{proof}
        \end{sol}
\end{enumerate}

\question For $x \in \R$ we have,
\[
  f(x) = \frac{x^4}{4} - \frac{x^3}{3}
\]
Give,
\[
  \min_xf(x) = \min_x\paren{\frac{x^4}{4} - \frac{x^3}{3}}
\]
Plot $f$ to double check your answer.

% solution goes here
\begin{sol}
  the global minimum of $f(x) = -\frac{1}{12}$ occurs at $x = 1$
\end{sol}

\question Let $x \sim \mathcal{N}(0, 1)$ be a normally distributed random
variable with mean 0 and standard deviation 1. What is the
likelihood that $x=\sqrt{2}$. Give an exact answer.

% solution goes here
\begin{sol}
  \[
  f(x)=\frac{1}{\sigma\sqrt{2\pi}}e^{-\frac{1}{2}(\frac{x-\mu}{\sigma})^2}
  \]
  \[
  f(x)=\frac{1}{1\sqrt{2\pi}}e^{-\frac{1}{2}(\frac{x-0}{1})^2} = \frac{1}{\sqrt{2\pi}}e^\frac{-x^2}{2}
  \]
  \[
  f(\sqrt{2}) = \frac{1}{\sqrt{2\pi}}e^{-\frac{2}{2}} = \frac{1}{\sqrt{2\pi}}e^{-1}= \frac{1}{e\sqrt{2\pi}}
  \]
  or 0 idk. I don't think x can be any specific value so it's 0 I think, but also the answer from before is the height of the graph at $f(\sqrt{2})$
\end{sol}

\question The code snippet in listing \ref{read} reads the plain text of the
novel \textit{The Cosmic Computer} by the famous American science
fiction writer Henry Beam Piper over the internet and saves it to
a string variable \texttt{text}.

\begin{lstlisting}[language=python, label=read, caption={A Python program to download the text of the \textit{The Cosmic Computer} by Piper.}]
from urllib.request import urlopen as get

url = 'https://www.gutenberg.org/files/20727/20727.txt'
with get(url) as response:
    text = response.read().decode('utf-8')

print(text)
\end{lstlisting}

Write a Python program that prints out the ten most used words in
the novel that have more than 5 letters. State your findings. Put
your code in a file called \texttt{frequency.py}.

% solution goes here
\begin{sol}
  maxwell: 86  \\
project: 76 \\
anything: 74 \\ 
something: 72 \\
storisende: 65 \\ 
poictesme: 61 \\
everybody: 61 \\
jacquemont: 59 \\
command: 58 \\
started: 57

\end{sol}

\question Watch \href{https://www.youtube.com/watch?v=5jwV3zxXc8E}{Laziness
  in Python - Computerphile} on YouTube.

Implement the function \texttt{fibonacci(n)} in the code listing
\ref{fib}.

\begin{lstlisting}[language=python, label=fib, caption={A Python 3 program to print out the first $n \geq 0$ Fibonacci numbers.}]
def fibonacci(n):
    # implement me

for t in fibonacci(50):
    print(t)
\end{lstlisting}

Run your program for $n=50$. What are the last five numbers
printed? Put your code in a file called \texttt{fibonacci.py}.

% solution goes here
\begin{sol}
  1134903170, 1836311903, 2971215073, 4807526976, 7778742049
\end{sol}

\section*{Submission Instructions}

\begin{enumerate}
  \item Submit a PDF that answers all the questions.
  \item Submit Python files \eg \texttt{frequency.py} and
        \texttt{fibonacci.py}.
\end{enumerate}

\end{document}

